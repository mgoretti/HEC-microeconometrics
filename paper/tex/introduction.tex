% !TEX encoding = UTF-8 Unicode


Awareness and discussion about obesity have been steeply rising in the last 40 years,  alongside the affected population. This subject is particularly relevant in the US which makes it the perfect place for studies.

Depression has also been the subject of recent discussions, on one hand regarding its role and prevalence in society to encourage better acceptance and also, on the other hand, about the recent results regarding its link with neuroinflammatory diseases (\cite{neuro}).

Several studies found that depression and obesity were linked, such as \cite{markowitz}, which proposed a bidirectional model and suggested treating them as a comorbid condition for health care purposes, \cite{luppino}, which confirmed a reciprocal link through a meta-analysis and \cite{dave} which found a seven percentage points increase in the probability of being overweight or obese in women with current or past depression diagnosis (no significant effect in men) and linked this causality to an increase of the economic burden of depression by about 10\% (9.7 billion \$).

Regarding the different prevalence of depression between sexes, \cite{silver} found that somatic symptoms of depression were much higher among women. Obesity could be classified in the later category.

This paper aims at confirming the previously cited effect of depression on obesity through the use of the depression of the spouse as an instrument for the endogenous depression of the subject.
