In this paper we showed that depression has a positive effect on obesity. When also considering the interaction effect with the gender, the cumulative effect was still significant in both sexes but was way stronger in women for any weight problem besides morbid obesity, on which depression has an equal positive effect for both sexes. These findings are in line with \cite{markowitz}, \cite{luppino} and \cite{dave} while also confirming the higher manifestation of somatic symptoms of depression in women found by \cite{silver}.

The novelty of the presented results consists in the use of the spouse's depression as an instrument for the subject's depression to correct the simultaneity bias and also finding a negative effect of depression on being overweight in males.

It must be noted that the obtained results are restricted to the special observed population which is composed of US residents with an over representation of older persons and also only applies to married individuals because of the needed instrument.